\documentclass{article}
\usepackage[utf8]{inputenc}
\usepackage{amsmath}


\title{Transfer Entropy between Major Cryptocurrency Pairs}
\author{Guy Van-Dam, Tom Bekor}
\date{August 2021}

\begin{document}

\maketitle

\section{abstract}
\section{Introduction}

*** Explanation about Cryptocurrencies *** 
% Cryptocurrencies are a kind of digital assets based on crypto- graphic protocols and technologies, such as the blockchain, that run on decentralized networks and make transactions secure and difficult to fake

Despite the rapidly increasing amount of \emph{altcoins} (short for alternative-coins) on the cryptocurrencies market, \emph{BTC} and \emph{ETH} are still the leading coins in a terms of market dominance of 43.6\% for \emph{BTC} and 18\% for \emph{ETH} according to \emph{coinmarketcap} *** at the time of writing this paper. link to the site (?)***

With this significant market share, said coins has a measurable impact of the other \emph{altcoins} price movement. We set out to quantify said impact by calculating the \emph{Transfer Entropy} between different coin pair on different *** candle sizes ***.

Transfer Entropy is defined as

$T_{X\rightarrow Y} = H\left( Y_t \mid Y_{t-1:t-L}\right) - H\left( Y_t \mid Y_{t-1:t-L}, X_{t-1:t-L}\right)$ 

where $H\left(A \mid B \right)$ is the \emph{Conditional Entropy} defined as 

$H\left(Y \mid X \right) = -\sum_{x\in\mathcal X, y\in\mathcal Y}p(x,y)\log \frac {p(x,y)} {p(x)}$

Using the Chain Rule for \emph{Conditional Entropy}

\begin{equation} \label{eq2}
\begin{split}
H(Y|X) &= \sum_{x\in\mathcal X, y\in\mathcal Y}p(x,y)\log \left(\frac{p(x)}{p(x,y)} \right) \\[4pt]
 &= \sum_{x\in\mathcal X, y\in\mathcal Y}p(x,y)(\log (p(x)) - \log (p(x,y))) \\[4pt]
 &= -\sum_{x\in\mathcal X, y\in\mathcal Y}p(x,y)\log (p(x,y)) + \sum_{x\in\mathcal X, y\in\mathcal Y}{p(x,y)\log(p(x))} \\[4pt]
 & = H(X,Y) + \sum_{x \in \mathcal X} p(x)\log (p(x) ) \\[4pt]
 & = H(X,Y) - H(X).
\end{split}
\end{equation}

We get a formula for the \emph{Transfer Entropy}

\begin{equation} \label{equation3}
\begin{split}
T_{X\rightarrow Y} &= H\left( Y_t \mid Y_{t-1:t-L}\right) - H\left( Y_t \mid Y_{t-1:t-L}, X_{t-1:t-L}\right) \\[4pt]
 &= H(Y_{t-1:t-L}, Y_t) - H(Y_{t-1:t-L}) - (H(Y_{t-1:t-L}, X_{t-1:t-L}, Y_t) - H(Y_{t-1:t-L}, X_{t-1:t-L})) \\[4pt]
 &= H(Y_{t:t-L}) - H(Y_{t-1:t-L}) - H(Y_{t:t-L}, X_{t-1:t-L}) + H(Y_{t-1:t-L}, X_{t-1:t-L}) \\[4pt]
 &= H(Y_t, Y_{t-1}, ..., Y_{t-L}) - H(Y_{t-1}, Y_{t-2}, ..., Y_{T-L}) - H(Y_t, Y_{t-1}, ..., Y_{t-L}, X_{t-1}, X_{t-2},..., X_{t-L}) + H(Y_{t-1}, Y_{t-2}, ..., Y_{t-L}, X_{t-1}, X_{t-2},..., X_{t-L})
\end{split}
\end{equation}



*** should we explain about candles? ***

*** add explanation about abnormality - explain about ATR ***

\section{Methods}

\subsection{Results}
\subsection{Discussion}

\end{document}
